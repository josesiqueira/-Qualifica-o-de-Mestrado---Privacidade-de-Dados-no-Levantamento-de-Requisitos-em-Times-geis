%     https://pt.wikihow.com/Resumir-um-Artigo-Cient%C3%ADfico
% Aqui eu vou fazer resumos dos artigos científicos que forem filtrados, separados por temas (as vezes os temas tem interseções!!!!!)
% Lembre-se: Aqui são resumos de artigos científicos! Para o BACKGROUND

%%%%%%%%%%%%%%%%%%%%%%%%%%%%%%%%%%%%%%%%%%%%%%%%%%%%%%%%%%%
%%%%%%%%%%%%%%%%%%%%%%%%%%%%%%%%%%%%%%%%%%%%%%%%%%%%%%%%%%%
Tema 1 - Privacidade
%%%%%%%%%%%%%%%%%%%%%%%%%%%%%%%%%%%%%%%%%%%%%%%%%%%%%%%%%%%
%%%%%%%%%%%%%%%%%%%%%%%%%%%%%%%%%%%%%%%%%%%%%%%%%%%%%%%%%%%

1- Recommender-based Privacy Requirements Elicitation - EPICUREAN: An Approach to Simplify Privacy Settings in IoT Applications with Respect to the GDPR

EPICUREAN: leia-se: Abordagem de Elicitação de Requisitos de Privacidade Baseado em recomendação.

PATRON é um sistema de privacidade para aplicações IoT que maximiza as funcionalidades de uma aplicação enquanto diminuem os dados que são divulgados. É um ambiente de execução sensível à privacidade. Entretanto, nele falta uma interface de usuário simples, enquanto é necessário especialistas no domínio para configurá-lo.

O EPICUREAN automatiza o processo de configuração do PATRON oferencendo sugestões de configurações de privacidade sob medida para as necessidades dos usuários.
Neste artigo, eles apresentam o EPICUREAN como uma extensão do PATRON, entretanto podendo ser usado fora deste (em um sistema de privacidade arbitrário). O Caso de uso é em uma aplicação Smart Health real. São vários os dados sensíveis do usuário que, de acordo com a GDPR, não podem ser divulgados. Smarth Health procura por padrões nos dados dos sensores de um wearable do paciente. Se um padrão é deectado então um serviço integrado de saúde é notificado. É transparente o motivo destes dados estarem em conformidade com a nova lei de Proteção de Dados da União Europeia, a GDPR. O Smart Health System para COPD (Doença Pulmonária Obstrutiva Crônica), chamado ECHO, tem acesso permanente aos dados dos sensores do paciente/data subject, como localização e as atividades que ele está fazendo (pois o Sistema sabe os padrões de atividade do paciente, para não dar falso positivo por exemplo), além disso pode derivar conhecimento a partir destes dados. (não tá no artigo:Imagina-se um atacante, que em posse destes dados, possa de maneira maliciosa se beneficiar da fragilidade de saúde para oferecer serviços mais caros, ou até mesmo oferecer o remédio errado, agravando o estado de saúde do paciente/data subject. E, sabendo qual atividade está fazendo, sabe se não está em casa).

A dificuldade do PATRON tá em que o usuário descreve em linguagem natural os requisitos de privacidade, e um especialista no domínio interpreta tal descrição.

Os autores expõem 6 especificação de requisitos de privacidade que foram encontrados na literatura (em 6 artigos distintos). Relatam que o PATRON não está em conformidade com eles, e apresentam o EPICUREAN.  


EPICUREAN: Processo de elicitação de requisitos de privacidade em três fases.
1ª fase:

2ª fase:

3ª fase:



%%%%%%%%%%%%%%%%%%%%%%%%%%%%%%%%%%%%%%%%%%%%%%%%%%%%%%%%%%%
2- Threat Poker: Solving Security and Privacy Threats in Agile Software Development




%%%%%%%%%%%%%%%%%%%%%%%%%%%%%%%%%%%%%%%%%%%%%%%%%%%%%%%%%%%
3- GDPR-Based User Stories in the Access Control Perspective





%%%%%%%%%%%%%%%%%%%%%%%%%%%%%%%%%%%%%%%%%%%%%%%%%%%%%%%%%%%
4- “Appropriate technical and organizational measures”: Identifying privacy engineering approaches to meet GDPR requirements



%%%%%%%%%%%%%%%%%%%%%%%%%%%%%%%%%%%%%%%%%%%%%%%%%%%%%%%%%%%
5- 






%%%%%%%%%%%%%%%%%%%%%%%%%%%%%%%%%%%%%%%%%%%%%%%%%%%%%%%%%%%
%%%%%%%%%%%%%%%%%%%%%%%%%%%%%%%%%%%%%%%%%%%%%%%%%%%%%%%%%%%
Tema 2 - Design Thinking
%%%%%%%%%%%%%%%%%%%%%%%%%%%%%%%%%%%%%%%%%%%%%%%%%%%%%%%%%%%
%%%%%%%%%%%%%%%%%%%%%%%%%%%%%%%%%%%%%%%%%%%%%%%%%%%%%%%%%%%
1- Aplicando Design Thinking em Engenharia de Software: um Mapeamento Sistemático

Definição: pode ser definido como uma metodologia utilizada por designers para abordar problemas, sendo aplicada em todas as áreas do conhecimento a fim de se obter inovação

%%%%%%%%%%%%%%%%%%%%%%%%%%%%%%%%%%%%%%%%%%%%%%%%%%%%%%%%%%%
%%%%%%%%%%%%%%%%%%%%%%%%%%%%%%%%%%%%%%%%%%%%%%%%%%%%%%%%%%%
Tema 3 - Agile Software Development
%%%%%%%%%%%%%%%%%%%%%%%%%%%%%%%%%%%%%%%%%%%%%%%%%%%%%%%%%%%
%%%%%%%%%%%%%%%%%%%%%%%%%%%%%%%%%%%%%%%%%%%%%%%%%%%%%%%%%%%
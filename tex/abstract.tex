\todo[inline]{(REVISADO) Só faça o abstract no final quando o resumo estiver validado.}\textit{
%Requirements management is now a common reality and an integral part of the evolution of the process of developing IT systems and solutions, especially since 2001, the year in which the agile manifesto was published, which is a declaration of principles that underpin the development agile software. Through the agile manifesto, a new approach was initiated regarding the development of systems, which defined a relation of priority between the principles addressed in the manifesto and some characteristics of traditional development
}
\textit{
%This work aims to propose a new approach regarding the management of requirements, especially the requirements of the agile methodology, investigating the requirements management models currently adopted by the companies and analyzing the context of the strategies proposed in view of the traditional view of management requirements. In addition, this work aims to perform a research with organizations that have a well defined requirements management process, seeking to identify the customer's expectations with the new requirements management approach proposed by this work. The focus of this research is to seek results that can help improve the quality of the requirements raised by the analyst by providing support to develop a suitable tool to support the clear understanding of customer needs.
}
\textit{
%The proposal of this new approach is based on the idea that the client is often unable to express his real needs of his own free will. In this case, it is necessary to stimulate and influence him so that he can express, in an adequate and transparent way, the problems for which he expects the solution to be proposed by the analyst to solve. For this purpose, some techniques can be used, such as Cognitive-Behavioral Therapy or Humanistic Therapy (more specifically, Client-Centered Therapy). According to Beck \ cite {beck}, Cognitive-Behavioral Therapy is based on understanding how people think (cognitive focus) and how they behave (behavioral focus), and Client-Centered Therapy, as stated by Rogers \ cite {rogers}, helps us understand our needs to enhance our personal growth based on three key aspects: empathy, unconditional positive acceptance, and authenticity.
}
\textit{
%The main contribution of this work is the proposal of a new approach to requirements management, oriented to creative thinking and focused on customer needs. The approach can be used by any company that wants to maximize efficiency in requirements elicitation, being adaptable and flexible, according to their needs.
}
Resumo ainda do Frederico, nao sei como escrever. Só depois que introdução tiver pronta.
\textbf{Contexto}: ....

\textbf{Objetivo}: É importante verificar, por meio de ....., se  uma organização .....;métricas e indicadores, se uma organização está executando seus processos de forma adequada, quais são os pontos de falha em um determinado projeto ou processo e se o alinhamento estratégico entre o negócio e a TI foi realizado de forma eficiente. Para isso, é importante que se definam indicadores que consigam fornecer a verificação da situação real do objeto a ser monitorado. Este trabalho tem como objetivo ......, investigando os ......, analisando o contexto das .... frente à visão .... adotada atualmente. Além disso, este trabalho visa realizar uma pesquisa junto as organizações que ....., buscando identificar a expectativa do cliente ...... 

\textbf{Metodologia}: Para..... foi utilizada a metodologia.... \textbf{Resultados}: Espera se com este trabalho obter  \textbf{Conclusão}: Espera-se concluir neste trabalho que 

% -- Enviado para Celia - SEMINARIOS : 09/03/2020 -> PARA 19/06/2020
No mundo atual e globalizado, os dados constituem uma fonte de riqueza, que deve ser protegido. A nova Lei Geral de Proteção de Dados brasileira entrará em vigor em agosto de 2020, e com ela, a necessidade de organizações desenvolverem software em conformidade com a nova lei, que trata da privacidade dos dados. As definições de Engenharia de Requisitos em times ágeis ainda é muito vaga, sem uma técnica bem definida para realizar a primeira fase, a de elicitação de requisitos. Design Thinking surge como uma maneira de propor soluções com um pensamento de um designer, valendo de times inclusivos, encorajando a criatividade e explorando novas ideias. Ao considerar os desafios que desenvolvedores de sistemas complexos enfrentam, além da necessidade de conformidade com leis e regulações, mais especificamente de privacidade, é interessante investigar como as técnicas de Design Thinking e Desenvolvimento de Software Ágil podem
Este trabalho espera auxiliar engenheiros de requisito a elicitarem, monitorarem e testarem sistemas de software para conformidade com a nova Lei Geral de Proteção de Dados brasileira, no contexto de times ágeis.
